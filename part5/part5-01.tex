\ifjp%>>>>>>>>>>>>>>>>>>>>>>>>>>>>>>>>>>>>>>>>>>>>>>>>>>>>>>>>>>>>>>>>>>>
\else%%%%%%%%%%%%%%%%%%%%%%%%%%%%%%%%%%%%%%%%%%%%%%%%%%%%%%%%%%%%%%%%%%%%
\fi%%%%<<<<<<<<<<<<<<<<<<<<<<<<<<<<<<<<<<<<<<<<<<<<<<<<<<<<<<<<<<<<<<<<<<

\part{\txt{\smlsharp{}の内部構造}{\smlsharp{} Internals and Data Structures}}
\label{part:internals}

\section*{\txt{はじめに}{Preface}}
\ifjp%>>>>>>>>>>>>>>>>>>>>>>>>>>>>>>>>>>>>>>>>>>>>>>>>>>>>>>>>>>>>>>>>>>>
	第\ref{part:internals}部では,\smlsharp{}コンパイラの内部構造を
詳述する.
	本部の目的は,本文書の第\ref{part:tutorial}部などを習得しML系関
数型言語の素養を持つ者が\smlsharp{}ソースコードの詳細を理解することである.
	読者としては,\smlsharp{}コンパイラの開発に加わろうとする者や
\smlsharp{}ソースコードを新しいコンパイラ開発等に利用しようとする者を主
に想定しているが,高水準プログラミング言語のコンパイラに興味を持つ一般の
読者にも参考となる文書となるように配慮した.

	オープンソースソフトウェアには尊敬すべき優れたコードが数多く存在
するが,それらの文化に接して筆者が感じる問題点の一つは,それら優れたコー
ドの構造や機能を,他の開発者や興味ある読者に理解できるように文書化する努
力が往々にして欠如していることである.
	コードそれ自体がドキュメントであるとの主張は,極端に規模の小さい
コードには当てはまるものの,数万行を超える大規模システムに対しては現実的
ではない.
	大規模システムでは,コード断片にはそのコード自身では理解不可能な
大域的な仮定や,他の複数のコード断片を制御するためのデータが含まれる.
	それらを理解するには,関連するシステム全般に関する処理の流れと,
その実現のためにシステムの各部分に分散してコード化されたデータの意味の全
体象を把握する必要がある.
	この大規模システムの複雑化に対処する現時点での唯一の方法は文書化
であると考える.
	さらに,コードの意図や構造を記述した文書は,それ自身,新たな構造
や機能の示唆を与えうる財産となると期待される.

	そこで,本文書では,その範を,筆者が尊敬するVAX/VMS OSの内部構造
の詳細な記述文書\cite{Kenah:1984:VID:225}に取り,\smlsharp{}コードが扱う
データ構造とコードの処理の詳細を,各機能が基礎とする理論やアルゴルズムと
共に記述することにした.
	本書が,\smlsharp{}のソースコードの開発や改良に取り組もうとする
者の理解の助けとなり,\smlsharp{}開発コミュニティ形成に資することを願う.

	第\ref{part:internals}部の構成は以下の通りである.
\begin{enumerate}
\item 第\ref{chap:package}章で\smlsharp{}ソースパッケージの構成を記述する.
\item 
      第\ref{chap:SimpleMain}章から第\ref{chap:LLVMCodegeneration}章までは,
\smlsharp{}コンパイラが操作するデータ構造とソースコードの詳細を,各フェー
ズ毎に記述する.
	コンパイルフェーズは通常複数のストラクチャから構成され,それら関
連するストラクチャは一つのディレクトリに置かれている.
	これら各章では,概ねディレクトリ単位でひとまとまりの処理を記述す
る. 
	補助的に使われているモジュールの機能を簡潔に記述した後,主要な処
理を行うモジュールの詳細を記述する.
\item 第\ref{chap:sqlintegration}章から第\ref{chap:interactivemode}章ま
での章では,フェーズにまたがる処理を記述する.
\item 第\ref{chap:bootstraping}章では,\smlsharp{}を拡張するためのブートスト
ラップ手順を記述する.
\item 第\ref{chap:runtimesystem}章では,\smlsharp{}の実行時処理系を記述
する.
\item 第\ref{chap:buildsystem}章では,\code{Makefile}の構成を含む,
\smlsharp{}コンパイラをコンパイルリンクするための環境を解説する.
\end{enumerate}
\else%%%%%%%%%%%%%%%%%%%%%%%%%%%%%%%%%%%%%%%%%%%%%%%%%%%%%%%%%%%%%%%%%%%%
	This part presents internals and data structures of the
\smlsharp{} compiler.
	The primiary readers are the developpers of the \smlsharp{}
compiler and all those who use the \smlsharp{} source code to extend and
modify the compiler. 
	To serve as a reference document on a full-scale compiler, which
can be read independenly, we also include the rationale, algorithms,
theories that underlie the compiler design and implementation. 

	Part \ref{part:internals} contains the following.
\begin{enumerate}
\item Chapter \ref{chap:package} describes the \smlsharp{} distribution
package.
\item 
      The chapters from Ch.\ref{chap:SimpleMain} to
Ch.\ref{chap:LLVMCodegeneration} describes the data structures and the
source code of each compilation phase in the order the compiler
processes  the source file.
\item 
	The chapters from Ch.\ref{chap:sqlintegration} to
Ch.\ref{chap:interactivemode} describes the processes that span
multiple compilation phases.
\item Chapter \ref{chap:bootstraping} describes the standard steps to
bootstrap the \smlsharp{} compiler.
\item 
	Chapter \ref{chap:runtimesystem} describes the runtime system.
\item 
	Chapter \ref{chap:buildsystem} describes data files such as
\code{Makefile} and various supporting scripts used to build the
\smlsharp{} system. 
\end{enumerate}
\fi%%%%<<<<<<<<<<<<<<<<<<<<<<<<<<<<<<<<<<<<<<<<<<<<<<<<<<<<<<<<<<<<<<<<<<

