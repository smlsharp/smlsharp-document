\chapter{\txt
{メイン関数コード生成モジュール:\code{GenerateMain}}
{The main function code generator : \code{GenerateMain}}
}
\label{chap:GenerateMain}

\ifjp%>>>>>>>>>>>>>>>>>>>>>>>>>>>>>>>>>>>>>>>>>>>>>>>>>>>>>>>>>>>>>>>>>>>
\begin{enumerate}
\item ソースロケーションとファイル構成
ディレクトリ \code{src/compiler/generatemain}下の以下のファイルからなる.
\begin{itemize}
\item \code{GenerateMain.sml} 処理本体
\item \code{GenerateMainError.ppg} エラーメッセージ定義
\end{itemize}

\item 機能概要 
\begin{enumerate}
\item 各オブジェクトファイルのメイン関数名の生成
\item 各オブジェクトファイルのエントリコード生成
\item トップレベルファイルに対するメイン関数呼び出しコードの生成
\end{enumerate}

\item 他モジュールとのインターフェイス
\begin{itemize}
\item \code{src/compiler/toplevel2/main/Top.sml}から呼び出される.
\end{itemize}
\end{enumerate}
\else%%%%%%%%%%%%%%%%%%%%%%%%%%%%%%%%%%%%%%%%%%%%%%%%%%%%%%%%%%%%%%%%%%%%
\fi%%%%<<<<<<<<<<<<<<<<<<<<<<<<<<<<<<<<<<<<<<<<<<<<<<<<<<<<<<<<<<<<<<<<<<

\section{\txt{\smlsharp{}プログラムのリンク方式}{Linking \smlsharp{} programs}}
\ifjp%>>>>>>>>>>>>>>>>>>>>>>>>>>>>>>>>>>>>>>>>>>>>>>>>>>>>>>>>>>>>>>>>>>>

	本モジュールを理解するためには,\smlsharp{}のリンク方式を理解す
る必要がある.
	第\ref{sec:SimpleMain.generateExec}節でも触れた通り,\smlsharp{}
のプログラムは宣言の列である.
	宣言の中でも変数の束縛宣言は,束縛対象の式を評価し値を計算し変数
をその値に束縛することによって実現する.
	\smlsharp{}の分割コンパイル機能を通じた別モジュールの変数の参照
は,この計算された値の参照である.
	\smlsharp{}の分割コンパイルを実現するためにには,分割コンパイル
された全てのモジュールの宣言文を評価,すなわちプログラムを実行する必要が
ある.

	\smlsharp{}コンパイラは,分割コンパイル対象プログラムに対して,
その変数宣言文を実行するための関数コードを生成する.
	この関数を,分割コンパイル単位のメイン関数と呼ぶ.
	分割コンパイルされた\smlsharp{}プログラムをリンクするには,変数
参照に関するアドレスを解決することに加えて,各コンパイル単位のメイン関数
を呼び出す必要がある.
	この実現のため,\smlsharp{}コンパイラは,分割コンパイルされたプ
ログラム集合をシステムリンカを呼び出してリンクし実行形式プログラムを作成
する前に,分割コンパイルされたプログラム集合のメイン関数を適当な順番で呼
び出すコードを生成し,このコードをプログラム全体のメイン関数として実行形
式プログラムにリンクする.
	この処理を実現するのが,メイン関数コード生成モジュールである.
	この処理実現のために,リンク対象のオブジェクトファイルのメイン関
数の名前を決める必要がある.
	コンパイラは,以下の方針でメイン関数名を決定している.
\begin{enumerate}
\item リンク対象オブジェクトファイルに対応するインターフェイスファイルに
対して,ハッシュ値を割り当てる.
	\code{AbsynInterface.interfaceName}型は,このハッシュ値を持つデー
タ構造である.
\item ハッシュ値に\code{\_main}文字列を付加し,メイン関数名とする.
\end{enumerate}

\else%%%%%%%%%%%%%%%%%%%%%%%%%%%%%%%%%%%%%%%%%%%%%%%%%%%%%%%%%%%%%%%%%%%%
\fi%%%%<<<<<<<<<<<<<<<<<<<<<<<<<<<<<<<<<<<<<<<<<<<<<<<<<<<<<<<<<<<<<<<<<<
	
\section{\txt{\code{GenerateMain.sml}ファイルの詳細}{The details of \code{GenerateMain.sml}}}
\ifjp%>>>>>>>>>>>>>>>>>>>>>>>>>>>>>>>>>>>>>>>>>>>>>>>>>>>>>>>>>>>>>>>>>>>

\subsection{インターフェイスファイル\code{GenerateMain.smi}}
\begin{program}
structure GenerateMain = \\
struct\\
\myem  val generate : AbsynInterface.interfaceName list -> YAANormal.topdecl list\\
\myem  val mainSymbol : AbsynInterface.compileUnit -> \{mainSymbol: string\}\\
\myem  val generateEntryCode\\
\myem\myem    : AbsynInterface.interfaceName list -> MachineCode.program -> MachineCode.program\\
end
\end{program}

\subsection{各関数の処理内容}
\begin{enumerate}
\item \code{mainSymbol}
	\code{interfaceName}型のデータを受け取り,メイン関数名を返す.
\item \code{generate}
リンク対象オブジェクトファイル集合に対応するインターフェイスファイ
ルの\code{interfaceName}型のリストを受け取り,そのメイン関数を順に呼び出
し\code{ANormal}中間コードを生成する.
\item \code{generateEntryCode}
リンク対象オブジェクトファイル集合に対応するインターフェイスファイ
ルの\code{interfaceName}型のリストを受け取り,そのメイン関数を順に呼び出
し\code{LLVM}コードを生成する.
\end{enumerate}

\subsection{使用するデータ構造と補助関数}

	NGブランチとLLVMブランチの処理が異なるため,後ほど執筆.

\else%%%%%%%%%%%%%%%%%%%%%%%%%%%%%%%%%%%%%%%%%%%%%%%%%%%%%%%%%%%%%%%%%%%%
\fi%%%%<<<<<<<<<<<<<<<<<<<<<<<<<<<<<<<<<<<<<<<<<<<<<<<<<<<<<<<<<<<<<<<<<<


