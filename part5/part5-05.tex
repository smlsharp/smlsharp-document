\chapter{\txt
{抽象構文木データ構造:\code{absyn}}
{Abstract syntax tree : \code{absyn}}
}
\label{chap:Absyn}

\ifjp%>>>>>>>>>>>>>>>>>>>>>>>>>>>>>>>>>>>>>>>>>>>>>>>>>>>>>>>>>>>>>>>>>>>
\begin{enumerate}
\item ソースロケーション \code{src/compiler/absyn}
\item 機能概要 ソース言語の抽象構文木を表現するデータ型
\item 処理概要
\begin{enumerate}
\item 構文木を表すデータ型の定義
\item 構文木プリンタの定義
\end{enumerate}
\item 他モジュールとのインターフェイス
\begin{enumerate}
\item \code{src/compiler/parser2/main/iml.grm} 文法の属性
\item \code{src/compiler/elaborate/main/Elaborator.sml} 構文論的解釈器への入力
\end{enumerate}
\end{enumerate}
\else%%%%%%%%%%%%%%%%%%%%%%%%%%%%%%%%%%%%%%%%%%%%%%%%%%%%%%%%%%%%%%%%%%%%
\begin{enumerate}
\item Location
	 \code{src/compiler/absyn}
\item Functionality
	A datatype defintion for the abstract syntax tree for the source
language.
\item Task Overview.
\begin{enumerate}
\item Datatype definition
\item Pretty printer definitons
\end{enumerate}
\end{enumerate}
\fi%%%%<<<<<<<<<<<<<<<<<<<<<<<<<<<<<<<<<<<<<<<<<<<<<<<<<<<<<<<<<<<<<<<<<<

\section{\txt{\code{absyn}モジュールの処理の詳細}{The details of \code{absyn} module}}

\ifjp%>>>>>>>>>>>>>>>>>>>>>>>>>>>>>>>>>>>>>>>>>>>>>>>>>>>>>>>>>>>>>>>>>>>
	構文木データ構造ディレクトリ\code{src/compiler/absyn/main/}
を構成するファイルの機能を記述する.

% ABSYN.sig*
% ABSYN_FORMATTER.sig*
% Absyn.ppg*
% Absyn.ppg.smi
% AbsynFormatter.smi
% AbsynFormatter.sml*
% AbsynInterface.ppg
% AbsynInterface.ppg.smi
% AbsynSQL.ppg
% AbsynSQL.ppg.smi
% Fixity.smi
% Fixity.sml*
% FormatTemplate.ppg
% FormatTemplate.ppg.smi
% Symbol.ppg
% Symbol.ppg.smi

\subsection{\code{Absyn.ppg}}
\begin{enumerate}
\item 機能概要 SQL埋め込み構文を除く抽象構文木データの型定義
\item 処理概要 
	以下の型を定義.

\begin{tabular}{ll}
\code{constant} & 定数リテラル \\
\code{headerFormatComment} & 現在未使用\\
\code{definingFormatComment} & 現在未使用\\
\code{callingConvention} & C関数への呼び出し形式\\
\code{ffiAttributes} & C関数の呼び出し属性.EXPFFIAPPLYのstring引数の解釈の値\\
\code{globalSymbolKind} & 単元型.現時点で意味はない.\\
\code{eq} & 型変数の同値性属性\\
\code{ty} & 構文論的な型\\
\code{tvarKind} & 型のカインド\\
\code{tvar} & 型変数\\
\code{ffiTy} & C関数の型表現\\
\code{pat} & パターン\\
\code{patrow} & レコードパターンのフィールド\\
\code{exbind} & 例外定義と例外エイリアス定義\\
\code{typbind}  & 型のエイリアス定義\\
\code{datbind}  & 再帰的データ型定義\\
\code{exp} & 式\\
\code{ffiArg}  & \code{\_ffiapply}式の引数\\
\code{dec}  & 宣言\\
\code{strdec}  & ストラクチャ宣言\\
\code{strexp}  & ストラクチャ式\\
\code{strbind}  & ストラクチャ宣言のエントリ\\
\code{sigexp}  & シグネチャ\\
\code{spec} & シグネチャのエントリ\\
\code{funbind}  & ファンクター式\\
\code{topdec}  & トップレベル宣言\\
\code{top}  & トップレベル構文\\
\code{interface}  & インターフェイス宣言\\
\code{unit} & コンパイル単位\\
\code{unitparseresult} &  コンパイル結果
\end{tabular}
\end{enumerate}
	
\subsection{\code{AbsynFormatter.sml}}
\begin{enumerate}
\item 機能概要 \code{Absyn.ppg}で定義した型のプリンタを提供
\end{enumerate}

\subsection{\code{AbsynInterface.ppg}}
\begin{enumerate}
\item 機能概要 インターフェイスファイルのデータ型定義
\item 処理概要 

以下の型を定義

\begin{tabular}{ll}
\code{constraint} & シグネチャ定義が透明化不透明化の区別\\
\code{overloadInstance} & オーバロード変数のインスタンス\\
\code{overloadCase} & オーバロード変数の型\\
\code{valbindBody} & \code{val}宣言のバリアント\\
\code{valbind} & \code{val}宣言\\
\code{typbind} & \code{type}宣言\\
\code{conbind} & \code{datatype}宣言のコンストラクタ\\
\code{datbind} & \code{datatype}宣言\\
\code{exbind} & 例外宣言\\
\code{idec} & インタフェイス宣言\\
\code{istrexp} & ストラクチャ本体\\
\code{strbind} & ストラクチャ宣言\\
\code{sigbind} & シグネチャ宣言\\
\code{funParam} & ファンクタパラメタ\\
\code{funbind} & ファンクタ宣言\\
\code{fixity}  & \code{infix}宣言のバリアント\\
\code{itopdec} & インタフェース宣言トップレベル\\
\code{itop}  & インターフェイスファイルエントリ(\code{\_require}または\code{\_include})\\
\code{filePlace} & ファイルパスモード\\
\code{source} & インターフェイスファイル名または\code{GENERATED}(\code{\_include}文)\\
\code{interfaceName} & インターフェイスファイル名とハッシュ値\\
\code{interfaceDec} & \code{\_require}されたインターフェイス宣言\\
\code{interface} & インターフェイス全体\\
\code{compileUnit} & ソース全体のコンパイル単位
\end{tabular}
\end{enumerate}

\subsection{\code{AbsynSQL.ppg}}
\label{sec:AbsynSQL}
\begin{enumerate}
\item 機能概要 埋め込みSQL言語の構文
\item 処理概要 
\begin{enumerate}
\item SQL言語の構文の型を多相型\code{('exp, 'pat, 'ty) exp}として定義す
る.
\item \code{Absyn.ppg}の\code{exp}型のバリアント
\begin{program}
\code{EXPSQL of (exp, pat, ty) AbsynSQL.exp * loc}
\end{program}
にSQL言語全体を埋め込む.
\end{enumerate}
	この構造は,\code{Absyn}型の一部である\code{AbsynSQL}構文の処理
をモジュラーに実現するために,埋め込むSQL言語に含まれる\smlsharp{}の要素
\code{exp}型,\code{pat}型,および\code{ty}型は,型パラメタとして実現す
る.
	したがっって,多相型\code{('exp, 'pat, 'ty) exp}は
\code{(Absyn.exp, Absyn.pat, Absyn.ty) exp}として使用される.
\end{enumerate}

\subsection{\code{Fixty.ppg}}
\begin{enumerate}
\item 機能概要 \code{infix}宣言のデータ構造の定義
\item 処理概要 
\begin{enumerate}
\item \code{infix}宣言の型\code{fixity}の定義
\item 組込中置演算子の定義\code{initialFixEnv}
\end{enumerate}
\end{enumerate}

\subsection{\code{FormatTemplate.ppg}}
\begin{enumerate}
\item 機能概要 \code{smlformat}の機能を内在化するためのフォーマットコメ
ント定義.
	現在未使用
\item 処理概要 
以下の型を定義

\begin{tabular}{ll}
\code{template} & フォーマットテンプレート\\
\code{instance} & フォーマッターインスタンス\\
\code{typepat} & フォーマットテンプレート定義のための型パターン\\
\code{formattag} & トップレベルフォーマットコメント
\end{tabular}
\end{enumerate}

\subsection{\code{Symbol.ppg}}
\begin{enumerate}
\item 機能概要 ソース言語の位置情報付きの識別子
\item 処理概要 
以下の型を定義
\begin{enumerate}
\item \code{symbol}
\item \code{longsymbol}
\end{enumerate}
\end{enumerate}
\else%%%%%%%%%%%%%%%%%%%%%%%%%%%%%%%%%%%%%%%%%%%%%%%%%%%%%%%%%%%%%%%%%%%%
\fi%%%%<<<<<<<<<<<<<<<<<<<<<<<<<<<<<<<<<<<<<<<<<<<<<<<<<<<<<<<<<<<<<<<<<<

