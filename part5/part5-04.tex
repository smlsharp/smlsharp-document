\chapter{\txt
{コンパイラトップモジュール:\code{toplevel2}}
{The compiler top module : \code{toplevel2}}
}
\label{chap:Loadfile}

\ifjp%>>>>>>>>>>>>>>>>>>>>>>>>>>>>>>>>>>>>>>>>>>>>>>>>>>>>>>>>>>>>>>>>>>>
\begin{enumerate}
\item ソースロケーションとファイル構成

ディレクトリ \code{src/compiler/toplevel2/main}下の以下のファイルからなる.
\begin{itemize}
\item \code{Top.sml} コンパイラトップレベルの処理を実現
\item \code{TopData.ppg} トップレベル処理のためのデータ構造の定義.
\item \code{NameEvalEnvUtils.sml} 対話型モード実現のため.名前評価モジュール環境の更新.
\end{itemize}

\item 機能概要 
\begin{enumerate}
\item ソースファイルのコンパイル
\item メイン関数コードの生成
\item 対話型環境の構築
\item 組込環境のセットアップ
\end{enumerate}
\item 他モジュールとのインターフェイス
\begin{itemize}
\item \code{src/compiler/main/main/SimpleMain.sml}から呼び出される.
\item \code{src/compiler/main/main/RunLoop.sml}から呼び出される.
\end{itemize}
\end{enumerate}
\else%%%%%%%%%%%%%%%%%%%%%%%%%%%%%%%%%%%%%%%%%%%%%%%%%%%%%%%%%%%%%%%%%%%%
\fi%%%%<<<<<<<<<<<<<<<<<<<<<<<<<<<<<<<<<<<<<<<<<<<<<<<<<<<<<<<<<<<<<<<<<<

\section{\txt{\code{Top.sml}の処理の詳細}{The details of \code{Top.sml}}}
\ifjp%>>>>>>>>>>>>>>>>>>>>>>>>>>>>>>>>>>>>>>>>>>>>>>>>>>>>>>>>>>>>>>>>>>>

	コンパイラトップモジュール機能を実現するファイルであり,\code{Top}
ストラクチャを定義する.

\subsection{インターフェイスファイル\code{Top.smi}}
\begin{program}
structure Top =\\
struct\\
\myem   datatype stopAt = datatype TopData.stopAt\\
\myem   datatype code = datatype TopData.code\\
\myem   datatype result = datatype TopData.result\\
\myem   type toplevelOptions = TopData.toplevelOptions\\
\myem   type toplevelContext = TopData.toplevelContext\\
\myem   type newContext = TopData.newContext\\
\myem   val defaultOptions : toplevelOptions\\
\myem   val emptyNewContext : newContext\\
\\
\myem   val extendContext : toplevelContext * newContext -> toplevelContext\\
\myem   val compile \\
\myem\myem       : toplevelOptions -> toplevelContext -> Parser.input -> LoadFile.dependency * result\\
\myem   val generateMain\\
\myem\myem       : toplevelOptions -> toplevelContext -> AbsynInterface.interfaceName list -> result\\
\myem   val loadInteractiveEnv \\
\myem\myem       : \{stopAt: stopAt,\\
\myem\myem\myem\           stdPath: Filename.filename list,\\
\myem\myem\myem\          loadPath: Filename.filename list\}\\
\myem\myem\myem\          -> toplevelContext -> Filename.filename -> newContext\\
\myem   val loadBuiltin : Filename.filename -> toplevelContext\\
end
\end{program}
\subsection{使用するデータ構造}
\code{TopData.ppg}で定義される以下の型を使用.
\begin{itemize}
\item \code{stopAt} コンパイラ起動スイッチに応じて以下の何れかのモードを
指定.
\begin{program}
SyntaxCheck | ErrorCheck | Assembly | NoStop
\end{program}
\item \code{code} 生成したオブジェクトコードのファイル名
\item \code{newContext} コンパイラが生成した名前環境.
\begin{program}
type newContext =\{topEnv: NameEvalEnv.topEnv, fixEnv: Elaborator.fixEnv\}
\end{program}
対話型モード実現に使用.
\item \code{result} コンパイル結果
\begin{program}
datatype result = STOPPED | RETURN of newContext * code
\end{program}
\item \code{toplevelOptions} コンパイルオプション
\item \code{toplevelContext} コンパイラが使用するコンテクスト
\begin{program}
type toplevelContext =\\
\myem  \{\\
\myem\  topEnv: NameEvalEnv.topEnv,\\
\myem\  version: int option,\\
\myem\  fixEnv: Elaborator.fixEnv,\\
\myem\  builtinDecls: IDCalc.icdecl list\\
\myem  \}
\end{program}
	\code{loadBuiltin}関数で生成され,対話型モードでは,前回のコンパ
イル結果が反映された環境が受け継がれる.
\end{itemize}
\subsection{各関数の処理詳細}
\begin{enumerate}
\item \code{extendContext}
	コンパイラのトップレベルコンテクストに,前回のコンパイルで得られ
たコンテクストをマージし,次のコンパイル単位のための環境を作成.
	対話型環境実現のために\code{RunLoop}で使用される.
\item \code{compile} 構文解析からコード生成に至る一連のコンパイルフェー
ズを実行し,結果の環境とオブジェクトコードを返す.
\item \code{generateMain}
	メイン関数コードを生成する.
\item \code{loadInteractiveEnv}
	対話型モードのための初期環境を構築.
\item \code{loadBuiltin}
	組込データ型の定義ファイルを読み込み,環境を構築.
\end{enumerate}

\else%%%%%%%%%%%%%%%%%%%%%%%%%%%%%%%%%%%%%%%%%%%%%%%%%%%%%%%%%%%%%%%%%%%%
\fi%%%%<<<<<<<<<<<<<<<<<<<<<<<<<<<<<<<<<<<<<<<<<<<<<<<<<<<<<<<<<<<<<<<<<<
