\chapter{\txt
{\code{IDCalc}データ構造}
{The \code{IDCalc} data structure}
}
\label{chap:IDCalc}

\chapter{\txt
{名前解析処理}
{Name Evaluation}
}
\label{chap:nameevaluation}

\chapter{\txt
{相互再帰関数最適化}
{Mutual recursive function optimization}
}
\label{chap:valrecoptimize}

\chapter{\txt
{\code{TypedCalc}データ構造}
{The \code{TypedCalc} data structure}
}
\label{chap:TypedCalc}

\chapter{\txt
{型推論}
{Type Inferene}
}
\label{chap:typeinference}

\chapter{\txt
{\code{RecordCalc}データ構造}
{The \code{Recordcalc} data structure}
}
\label{chap:RecordCalc}

\chapter{\txt
{パターンマッチングコンパイル処理}
{Pattern Matching Compilation}
}
\label{chap:matchcompilation}

\chapter{\txt
{外部関数のコンパイル処理}
{Foreign function Compilation}
}
\label{chap:fficompilation}

\chapter{\txt
{型主導レコードコンパイル}
{Type-Directed Record Compilation}
}
\label{chap:recorcdcompilation}

\chapter{\txt
{\code{TypedLambda}データ構造}
{The \code{TypedLambda} data structure}
}
\label{chap:TypedLambda}

\chapter{\txt
{データ型のコンパイル}
{Datatype Compilation}
}
\label{chap:datatypecompilation}

\chapter{\txt
{\code{BitmapCalc}データ構造}
{The \code{BitmapCalc} data structure}
}
\label{chap:BitmapCalc}

\chapter{\txt
{型主導ビットマップ生成処理}
{Type-Directed Bitmap Generation}
}
\label{chap:bitmapgeneration}

\chapter{\txt
{\code{ClosureCalc}データ構造}
{The \code{ClosureCalc} data structure}
}
\label{chap:ClosureCalc}

\chapter{\txt
{クロージャー変換}
{Closure Conversion}
}
\label{chap:closureconversion}

\chapter{\txt
{\code{RuntimeCalc}データ構造}
{The \code{RuntimeCalc} data structure}
}
\label{chap:RuntimeCalc}

\chapter{\txt
{型主導のコーリングコンベンション生成}
{Type-Directed Calling Convension Generation}
}
\label{chap:callingconventiongeneration}

\chapter{\txt
{\code{ANornal}データ構造}
{The \code{ANormal} data structure}
}
\label{chap:ANormal}

\chapter{\txt
{A正規化処理}
{A-Normalization}
}
\label{chap:anormalization}

\chapter{\txt
{\code{MachineCode}データ構造}
{The \code{MachineCode} data structure}
}
\label{chap:MachineCode}

\chapter{\txt
{抽象機械語生成}
{Abstract Machine Code Generation}
}
\label{chap:abstractcodegeneratrion}

\chapter{\txt
{フレームスタック割り当て}
{Stack Frame Allocation}
}
\label{chap:stackframeallocation}

\chapter{\txt
{\code{LLVMIR}データ構造}
{The \code{LLVMIR} data structure}
}
\label{chap:LLVMIR}

\chapter{\txt
{LLVMコード生成}
{LLVM Code Generation}
}
\label{chap:LLVMCodegeneration}

\chapter{\txt
{\smlsharp{}の組込データ型}
{Builtin Datatypes in \smlsharp{}}
}
\label{chap:bootstraping}

\section{概要}
\begin{enumerate}
\item 機能概要.
	コンパイラに組み込まれた型の定義と束縛を実現.
\item 処理概要
	\smlsharp{}コンパイラが,起動時に構成する組込み型の定義,その環
境への束縛,組込み方の属性取り出し等を実現するために,一連の定義ファイル
およびユーテリティ関数を用意.
\item 関連モジュール.型を扱うコンパイラのフェーズすべて.

\end{enumerate}

\section{処理の詳細}
\begin{enumerate}
\item \smlsharp{}における型の意味論的な実体.
	\smlsharp{}言語では,型は\code{int}などの組込型定数や
\code{datatype}などの構文で定義される.
	\smlsharp{}コンパイラにおけるこれら構文論的な型の意味論的実体は,
\code{TypID.id}型のユニークなIDである.
	このIDを持つ意味論的な型は\code{NameEval}ストラクチャで
\code{tfun}型の値として生成され,その表現が\code{InferTypes2}で
\code{tyCon}型の値に変換される.
	ただし,型の意味論であるIDは変化しない.

	意味論的な型は,コンパイラは\code{NameEval}おいて,構文論的型を
解釈するための環境の中に保持し,構文論的な型を意味論的な型に変換している.
	\code{NameEval}での環境は以下の構造である.
\begin{verbatim}
  type topEnv = {Env:env, FunE:funE, SigE: sigE}
  type env = ENV of {varE: IDCalc.varE, tyE: tyE, strE: strE}
  type tyE = tstr SymbolEnv.map
  datatype tstr
    =  TSTR of IDCalc.tfun
    |  TSTR_DTY of {tfun:IDCalc.tfun, varE:IDCalc.varE,
                    formals:IDCalc.formals, conSpec:IDCalc.conSpec}
\end{verbatim}


\item 組込型を扱う主なファイル・ストラクチャ.

	コンパイラは,組込型の実現のために,構文論的な組込型に対応する意
味論的な型を,コンパイラの環境にあらかじめセットする.
	そのための主なデータ構造は以下の通りである.

\begin{itemize}
\item \module{compiler/runtimetypes/main/RuntimeTypes.ppg}{RuntimeTypes}

	実行時の型の定義.
	組込型はすべて\code{runtimeTy}フィールドに実行時型を持つ.

\item \module{builtin.smi}{SMLSharp\_Builtin 他}

	コンパイラが使用するために,組込み型を束縛するインターフェイスファ
イル.
	このファイルを\code{NameEval.loadInterface}で読み込むことによっ
て組込み型が使用可能になる.
	コンパイラの\code{SimpleMain}の\code{loadBuiltin}で読み込まれる.

\item \module{compiler/builtin2/main/BuiltinTypeNames.ppg}{BuiltinTypeNames}

	コンパイラで組込型を扱うための名前としてのデータ構造を定義してい
る.

\item \module{compiler/builtin2/main/BuiltinTypes.sml}{BuiltinTypes}

	\code{BuiltinTypeNames}で定義された組込型の実体を定義.
	型は通常\code{NameEval}によって,ユーザの\code{datatype}宣言文から生
成される.
	ここで生成された型の識別子\code{TypID.id}は,コンパイルフェーズ
中受け継がれる.
	\code{BuiltinTypes}ストラクチャは,\code{NameEval}で生成される形
式の組込み型を直接生成している.
\end{itemize}

\item コンパイル対象言語の組込型の生成.

	以上のデータ構造を使い,コンパイラは,以下の手順で組込型を環境に
設定する.
\begin{enumerate}
\item コンパイラの起動時に,\code{BuiltinTypes}ストラクチャがロードされ,
その時の副作用として,組込型の意味論的実体が生成される.
	この実体は,各型毎に,\code{NameEval}で使用する\code{tstr}と,そ
れを変換し\code{InferTypes2}のための変換した\code{tyCon}として,インター
フェイスファイルを通じて外部に提供される.

\item コンパイラ起動時に,\code{SimpleMain}の\code{loadBuiltin}が
\code{builtin.smi}を読み込み,コンパイラの環境にロードしている.
	このファイルは,言語組込型を提供するため,オブジェクトレベルのス
トラクチャとして実現できない.
	そこで,\code{builtin.smi}は,対応するストラクチャを持たないイン
ターフェイスファイルとして実現し,\code{BuiltinType}で直接生成した
組込型の意味論的実体を直接参照する必要がある.
	そのための特殊構文として,
\begin{verbatim}
  datatype int  = _builtin datatype int
\end{verbatim}
を用意している.
	\code{loadBuiltin}から\code{Top.loadBuiltin}を経由して呼び出され
た\code{NameEval}の\code{evalBuiltin}は,この\code{\_builtin}とマークさ
れた名前\code{int}を,\code{BuiltinTypes}で直接定義された組込型と解
釈し,\code{BuiltinTypes}の\code{findTstrInfo}関数を用いて意味論的型
を取り出し,環境に設定する.
\end{enumerate}
\item 組込型の追加手順.

	以下の手順で組込型を\code{foo}追加できる.
\begin{enumerate}
\item \module{compiler/builtin2/main/BuiltinTypes.sml}{BuiltinTypes}
に\code{foo}の定義を追加.
\begin{enumerate}
\item 意味論的実体を直接生成するため,他の例にならい,
\code{foo}に必要な引数の適当に設定し\code{makeTfun}を呼び出す.
\item この結果を変数として束縛し,\code{NameEval}で参照するため,
\code{makeTfun}の呼出し結果を束縛する変数を{BuiltinTypes.smi}に追加する.
\item \code{NameEval}の\code{evalBuiltin}で束縛した変数を参照するため,
\code{findTstr}関数に\code{foo}のケースを追加.
\end{enumerate}
\item 以上のソースでコンパイラを再コンパイルする.
	再コンパイルによって作成されたコンパイラは,\code{BuiltinTypes}
に\code{foo}の意味論的型が追加されており,\code{foo}型を解釈することがで
きる.
\item \module{builtin.smi}{foo}に\code{foo}エントリを追加.
	再コンパイルしたコンパイラのソースの\code{builtin.smi}ファイルに
\code{foo}エントリを追加し,インストールする.
	これによって,このコンパイラは,\code{foo}を組込型として利用する
ソースファイルをコンパイル可能である.
\item \module{compiler/builtin2/main/BuiltinTypeNames.ppg}{BuiltinTypeNames}の
\code{dty}にも\code{foo}エントリを追加する.
	このステップは,その組込型が新しい実行時型である場合などに必要となる.
\end{enumerate}
	これ以降は,ユーザレベルで\code{foo}と書けば,組込型として解釈さ
れる.
	ただし,この変更後このコンパイラーソースは,そのソースツリーの
\code{minismlsharp}ではコンパイルできない.
	このコンパイラのコンパイルには,\code{SMLSHARP=<newsmlsharp>}と
して新しいコンパイラを指定する必要がある.
\end{enumerate}


\chapter{\txt
{SQL言語の埋め込み}
{SQL Integration}
}
\label{chap:sqlintegration}

\section{概要}
\begin{enumerate}
\item 機能概要.
	\smlsharp{}にSQL言語を多相型をもつ式としてシームレスに埋め込む.
\item 処理概要.
	SQLの型理論(SIGMOD89,TODS96)を用いて,SQLの埋め込み方式
(ICFP2011)に従い,以下の手順で,SQLをシームレスに埋め込む.
\begin{enumerate}
\item \smlsharp{}の構文木にSQL文法を追加.
\item パーザの文法規則の追加.
\item ElaborateフェーズでのSQL構文のコンパイル.
\item 型推論時に,存在型を導入し,SQL接続の一貫性を保証する.
\end{enumerate}
\item 他モジュールとのインターフェイス
\begin{enumerate}
\item AbsynへのSQL構文の埋め込み
\item 上記埋め込みの構文解析のたの\code{iml.grm}と\code{iml.lex}の変更.
\item EraborateCoreからElaborateSQLの呼び出しを追加.
\item 型推論モジュールにて\code{\_sqlserver}式の型を内部表現に変換.
\end{enumerate}
\end{enumerate}

\section{処理の詳細}
\begin{enumerate}
\item \smlsharp{}の構文木にSQL文法を追加.
	\code{Absyn}モジュールにケースエントリ\code{EXPSQL}を追加し,
この項に,SQL言語全体を埋め込む.
	SQL式内で\smlsharp{}式が使用出来るためSQL式と\smlsharp{}式は相互
に依存しあう構造をしている.
	SQLのモジュール性の高い埋め込みを実現するため,以下の構造により,
相互再帰構造を分解する.
\begin{verbatim}
  structure AbsynSQL = 
  struct
    datatype exp = ...
      | EXPSQL of (exp, pat, ty) AbsynSQL.exp * loc
      ...
  end
  structure AbsynSQL =
  struct
    datatype ('exp, 'pat, 'ty) exp =
    ...
  end
\end{verbatim}
	後者の型パラメタ\code{'exp},\code{'pat},\code{'ty}には,
\code{Absyn.exp},\code{Absyn.pat},\code{Absyn.ty}が代入される.
	この構造により,\code{AbsynSQL}を独立のファイルとして定義できる.
\item パーザの文法規則の追加.

	SQLと\smlsharp{}ではキーワード集合が異なる.
	ここでは,SQL構文では,SQLキーワドと\smlsharp{}キーワード双方を
キーワードとして扱い,\smlsharp{}の式では\smlsharp{}キーワードのみキーワー
ドとして扱い,SQLキーワードは通常の識別子として使用出来る仕様とする.
	Standard MLとの後方互換性を維持する上で必要である.
	この実現のため,文法定義を以下のように拡張する.
\begin{itemize}
\item SQLの{\tt SELECT}などのキーワードを終端記号として追加.
\item \smlsharp{}の識別子\code{id},\code{longid},原子式\code{atexp},
アプライ式\code{appexp}をそれぞれ,
\begin{enumerate}
\item SQLキーワードを含まないもの
(\code{id\_noSQL},\code{longid\_noSQL},\code{atexp\_noSQL},\code{appexp\_noSQL})
\item 従来の\smlsharp{}の要素(\code{id},\code{longid},\code{atexp},\code{appexp})
\end{enumerate}
に分割.
	SQL構文では,{\tt appexp\_noSQL}を使用し,それ以外の\smlsharp{}
構文では{\tt appexp}を使用する.
\end{itemize}
\item SQL構文のコンパイル.

	Elaborateフェーズで\code{AbsynSQL}で表現されるSQL構文を,
\code{SQL}用のライブラリ呼び出しを含む\smlsharp{}の式に変換する.
	この式は,ICFP2011の枠組みを用いて,
\begin{itemize}
\item SQL構文の結果の型を表現する(擬似)項
\item SQLサーバへ送信する文字列を生成する項
\end{itemize}
の2つの組みを生成する.
	この処理は\code{ElaboreteSQL}に定義され,\code{Absyn}の構造に従
い,\code{Absyn.exp}を翻訳する\code{ElaborateCore.elabExp}
から,\code{AbsynSQL.exp}の中に現れる
\code{Absyn.exp}と\code{Absyn.pat}を処理するための
\code{ElaborateCore.elabExp}と\code{ElaborateCore.elabPat}
とともに呼び出される.

\item \smlsharp{}の型とSQLの型の相互変換.
	SQLの型はRDBサーバーによって定義されている型であり,クライアント
側ではそれら値は文字列で表現される.
	SQLを\smlsharp{}に埋め込むには,それらSQL型を\smlsharp{}の型とし
て扱えなければならない.
	そのために以下の処理を行う.
\begin{itemize}
\item SQL組込み型と\smlsharp{}の型の対応の確立.
	SQLの組込み型は,対応する\smlsharp{}の型として扱う.
	\code{InferType}モジュールで,\code{isCompatibleWithSQL}関数が
\code{true}を返す型が,\smlsharp{}でSQLデータとしても扱うことができる型
であり,\code{\_sqlserver}式でカラムの型として宣言できる.
\item SQLサーバでの型名の記録
	コンパイラは,SQLと互換の型のSQLサーバ上での名前を記録している.
	この情報を使い,\code{SQL.connect}関数で動的型チェックされる.
	現在SQLサーバ上での名前は,\code{PGSQLBackend}の
\code{translateType}関数にハードコードされている.
	この関数を使って,SQLのカタログテーブルの型名が\code{SQLPrim}の
\code{connect}が型の対応をとり,動的型チェックがなされる.

\item SQL文字列表現と対応する\smlsharp{}の型間の相互変換.
	\code{SELECT}文の結果得られるSQL型の値はすべて文字列で表現されて
いる.
	\smlsharp{}でのSQL対応型は,SQLの問い合わせの生成時に文字列に変
換され,問い合わせの結果の取り出し関数で,文字列から逆変換される.
\begin{enumerate}
\item 問い合わせの結果の変換.
	問い合わせの結果は,\smlsharp{}では\code{'a rel}型のオブジェクト
である.
	この型は,
\begin{verbatim}
  val fetch : 'a rel -> ('a * 'a rel) option
\end{verbatim}
で{\tt 'a}型に変換される.
	この変換は,問い合わせ生成時に,結果の文字列からの変換関数をあら
かじめ用意することによって実現される.
	この変換関数は,\code{SQLPrim}で\code{fromSQL}として多重定義され
ている.
	そのインスタンスは,\code{SQLPrim}に型毎に定義されており,そこか
らサーバー依存の\code{Bckend}の\code{getXXX}が参照されている.
	サーバー依存の\code{Bckend}では,実際の文字列変換が行う
\code{getXXX}が定義されている.

\item SQL文字列への変換.

	SQL問い合わせは,\smlsharp{}でSQLとの互換型を使用して作成される.
	このSQL互換型$\tau$の値は,問い合わせ内部では
\code{($\tau$,$witness$)SQL.value}型として使用することが要求される.
	\code{($\tau$,$witness$) SQL.value}型は,内部にSQL文字列表現と
問い合わせの一貫性を保証する$witness$変数をもつ.
	$witness$の詳細は本文書では省略する.

	$\tau$型から\code{($\tau$,$witness$) SQL.value}型への変換は,
\code{SQL.smi}で定義された多重定義関数\code{toSQL}で行われる.
	そのインスタンス実体は\code{SQLPrim.sml}で定義されている.
	そのインスタンス実体の中で,各型を文字列に変換する関数が適用され
る.

\end{enumerate}
\end{itemize}
\end{enumerate}

\section{現状でのSQL互換型追加の流れ}
	SQLのより完全なサポートのためには,SQL型のより系統的な扱いが必要
である.
	その方針は,次節で述べる.
	ここでは,現状のコードにおいて新しいSQL互換型$T$互換型の
追加手順を述べる,\code{timestamp}における例を記述する.
\begin{enumerate}

\item 追加するSQL互換型$T$を組込み型としてコンパイラに追加する.

	「組込み型の定義と束縛」を参考に追加し,さらに,型推論ストラクチャ
\code{InferTypes2}の\code{isSQLBuiltinTy}に\code{true}エントリーを追加す
る.

	ただし,SQL互換型は数多く存在し,さらに,\smlsharp{}の型と類似の
名前も多い.
	このため,束縛の際には,統一的な名前とスコープの管理が課題である.
	当面,SQL互換型の束縛は以下の方針をとる.
\begin{itemize}
\item \code{builtin.smi}に\code{SMLSharp.SQL.SQLPrimitiveTyoes}ストラク
チャを定義し,そこに型の束縛を追加する.
\item \code{SQL.smi}に,トップレベルの束縛を追加する.
	従って,ユーザは,
\begin{enumerate}
\item \code{\_require "basis.smi"}
\item \code{\_require "sql.smi"}
\end{enumerate}
の順で記述すると,SQL互換型がトップレベルで使用可能となる.

	\code{timestamp}型の場合,\code{timestamp}型をコンパイラにを追加
し,\code{basis.smi}に
\begin{verbatim}
  structure SQLPrimitiveTypes =
  struct
    datatype timestamp = _builtin datatype timestamp
  end
\end{verbatim}
の束縛を追加し,\code{sql/main/SQL.smi}に
\begin{verbatim}
  type timestamp = SMLSharp.SQLPrimitiveTypes.timestamp
\end{verbatim}
の束縛を追加すれば実現できる.

\end{itemize}
\item サポートストラクチャの構築.

	追加したSQL互換型のための種々のプリミティブ関数をサポートするスト
ラクチャとインタフェイスファイルを\code{src/sql/builtintypes/$typeName$/}
以下に作成する.
	この中で,SQL文字列との相互変換のため
\begin{verbatim}
  val toString : timestamp -> string
  val fromString : string -> timestamp
\end{verbatim}
の2つは標準でサポートすることとする.

	\code{timestamp}の場合,\code{TimeStamp.sml}と
\code{TimeStamp.smi}を構築する.
	サポートライブラリ実現のためのC関数が必要であれば,それらを作成
し同一ディレクトリに置く.
	\code{timestamp}の場合,文字列との変換のために,
\code{string\_to\_time\_t.c}と\code{timeval\_to\_string.c}を作成した.

\item SQL組込み型と\smlsharp{}の型の対応の確立.

	\code{InferType}モジュールの\code{isCompatibleWithSQL}関数を拡張
し,\code{timestamp}型が\code{true}を返すように拡張する.

\item SQLサーバでの型名の記録

	\code{PGSQLBackend}の\code{translateType}関数を
\code{timestampe}文字列を\code{timestampe}型と解釈するように変更する.

\item SQL文字列表現と対応する\smlsharp{}の型間の相互変換.

	\code{timestamp}型が\smlsharp{}での対応する内部構造に変換される
よう,以下の定義を追加する.
\begin{enumerate}
\item 問い合わせの結果の変換.

	\code{SQLPrim}の\code{fromSQL}多重定義を拡張する.
	さらに,サーバー依存の\code{Bckend}に\code{get$T$}を追加.

\item SQL文字列への変換.

	\code{SQL.smi}で定義された多重定義関数\code{toSQL}を新しい互換型
$T$へ拡張し,そのインスタンス実体を\code{SQLPrim.sml}に定義する.

\item \code{SQLCompilation}における\code{compileColumn}の拡張

	
	SQLCompilationでの\code{\_sqlServer}式をコンパイル時に,サーバに
指定された各カラムを型チェックし,各型の式を作っている.
	そこで,SQL互換型に対して,\code{compileColumn}の型チェックケー
スを加え,さらに,\code{stringConst}などを例に,新しく加えたSQL互換型の
項を\code{RecordCalc}言語の項を生成する.
	
	問題点としては,型によっては定数などの簡単に生成する型が存在しな
い場合がある.
	そこで,この項の生成は,よりユニフォームに行うのがよいであろう.
	実際の定数は必要としないので,例えば\code{Cast 0 : $T$}等で十分
とおもわれる.
	この項が実行時に評価されるのであれば,実行時の型が一致している必
要があり,面倒である.
	評価しないようにすべきであろう.
\end{enumerate}
\end{enumerate}


\chapter{\txt
{オーバーロード演算子の処理}
{Primitive Operator Overloading}
}
\label{chap:primitiveoverloading}

\chapter{\txt
{対話型環境}
{Interactive mode}
}
\label{chap:interactivemode}

\chapter{\txt
{\smlsharp{}の実行時処理系}
{\smlsharp{} Runtime System }
}
\label{chap:runtimesystem}

\chapter{\txt
{\smlsharp{}のビルド環境}
{Building Environment of \smlsharp{}}
}
\label{chap:buildsystem}


