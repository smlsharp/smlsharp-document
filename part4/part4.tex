\part{\txt{プログラミングツール}{Programming Tools}}

\ifjp%>>>>>>>>>>>>>>>>>>>>>>>>>>>>>>>>>>>>>>>>>>>>>>>>>>>>>>>>>>>>>>>>>>>
\else%%%%%%%%%%%%%%%%%%%%%%%%%%%%%%%%%%%%%%%%%%%%%%%%%%%%%%%%%%%%%%%%%%%%
\fi%%%%<<<<<<<<<<<<<<<<<<<<<<<<<<<<<<<<<<<<<<<<<<<<<<<<<<<<<<<<<<<<<<<<<<


\chapter{\txt{構文解析器生成ツール smlyaccとsmllex}{Parser generator smlacc and smllex}}
\ifjp%>>>>>>>>>>>>>>>>>>>>>>>>>>>>>>>>>>>>>>>>>>>>>>>>>>>>>>>>>>>>>>>>>>>
	\smlsharp{}には構文解析器生成ツールsmlyaccが同梱されている.
	このツールはAndrew W. AppelとDavid R. Tarditiによって開発された
ソースコードのプログラムインターフェイスを単純化したものである.
	オリジナルなソフトウェアのライセンス情報は,smlyaccのソースディ
レクトリ\code{src/ml-yacc/COPYRIGHT}に,また,その使い方は
\code{src/ml-yacc/doc/mlyacc.tex}にある.

	ソースファイルの記述方法はオリジナルのものと同一である,
\code{mlyacc.tex}を参照されたい.
\else%%%%%%%%%%%%%%%%%%%%%%%%%%%%%%%%%%%%%%%%%%%%%%%%%%%%%%%%%%%%%%%%%%%%
\fi%%%%<<<<<<<<<<<<<<<<<<<<<<<<<<<<<<<<<<<<<<<<<<<<<<<<<<<<<<<<<<<<<<<<<<

\section{\txt{構文解析システム構造}{Structure of a Parser}}

\ifjp%>>>>>>>>>>>>>>>>>>>>>>>>>>>>>>>>>>>>>>>>>>>>>>>>>>>>>>>>>>>>>>>>>>>
	smlyaccはLRLR構文解析システム構築ツールである.
	ツールの利用には,通常以下のファイルを用意する.
	説明のため,言語
\begin{enumerate}
\item 
\end{enumerate}
	

\else%%%%%%%%%%%%%%%%%%%%%%%%%%%%%%%%%%%%%%%%%%%%%%%%%%%%%%%%%%%%%%%%%%%%
\fi%%%%<<<<<<<<<<<<<<<<<<<<<<<<<<<<<<<<<<<<<<<<<<<<<<<<<<<<<<<<<<<<<<<<<<
