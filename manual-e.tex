%#!platex manual-e; latexml --noparse --destination=./manual-e.xml manual-e.tex; latexmlpost --format=xhtml --destination=/home/ohori/work/WWW/pllab-www/smlsharp/docs/1.0/en/manual-e.xhtml --split  --novalidate manual-e.xml
\documentclass{book}
\usepackage{a4}
\usepackage{latexsym}
\usepackage{url}
\usepackage{epsfig}
\oddsidemargin 0in
\evensidemargin 0in
\date{April 6th, 2012}
\newcommand{\smlsharp}{SML\#}

\newcommand{\version}{1.00}
\newcommand{\smlsharpSize}{30万}
%%%%% set the following!
\newcommand{\authors}{}

\newcommand{\func}{\rightarrow}
\newcommand{\vbar}{\mbox{\ |\ }}
\newcommand{\code}[1]{\mbox{{\tt #1}}}
\newcommand{\nonterm}[1]{\mbox{$\langle$}{\it #1}\mbox{$\rangle$}}
\newcommand{\sep}{\mbox{\ \ }}

\newenvironment{program}{\begin{tt}\begin{quote}}{\end{quote}\end{tt}}
% \newcommand{\myem}{\ \ \ \ \ \ \ }
\newcommand{\myem}{\ \ \ \ \  }
\newcommand{\myfm}{ \ \ \ \ \ }


\title{The \smlsharp{} Language Document}
\author{
\authors
\\
RIEC, Tohoku University
}


\begin{document}
\maketitle

\tableofcontents

	This is the official documant of the programming language 
\smlsharp{} version \version, developed at RIEC, Tohoku Universit,
and released on April 6th, 2012.

\bigskip

\begin{Large}
The English documentain will be availabe soon. For the time being, see the Japanese version at 
\url{http://www.pllab.riec.tohoku.ac.jp/smlsharp/docs/1.0/ja/index.xhtml}
in case if you understand Japanese.
\end{Large}

% \part{はじめに}
\chapter*{Preface}

	This is the official documant of the programming language 
\smlsharp{} version \version, developed at RIEC, Tohoku Universit,
and released on April 6th, 2012.

\bigskip

\begin{Large}
The English documentain will be availabe soon. For the time being, see the Japanese version at 
\url{http://www.pllab.riec.tohoku.ac.jp/smlsharp/docs/1.0/ja/index.xhtml}
if you understand Japanese.
\end{Large}

\end{document}
